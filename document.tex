\documentclass[12pt,a4paper,twocolumn]{article}
\usepackage[utf8]{inputenc}
\usepackage[T1]{fontenc}
\usepackage{times}
\usepackage[left=1.5cm,top=1.2cm,right=1.5cm,bottom=1.2cm]{geometry}
\usepackage{amsmath}
\usepackage{amssymb}
\usepackage{makeidx}
\usepackage{graphicx}
\usepackage[portuguese]{babel}
\title{\textbf{Diodos aplicados à sensores digitais de imagem}}
\author{Leandro Assis dos Santos}

\begin{document}
\maketitle

\section*{Introdução}
		Sensores digitais de imagem são sensores que transdutam a intensidade luminona incidente sobre eles em sinais elétricos analizáveis por um sistema computacional. Sensores de imagem são usados em imagens eletrônicas aplicadas a diversos dispositivos como câmeras digitais, mouses ópticos, equipamentos de imagem médica, equipamentos de visão noturna e térmica, LIDARs (Light Detection and Ranging), dentre outros.
				
		Esses sensores são constituídos de uma matriz de pixels disposta sobre um painel plano. Cada pixel dessa matriz consiste de uma célula com um fotodiodo que pode estar associada à um transistor MOS a depender do tipo de sensor em questão. Existem dois principais tipos de sensores de imagem eletrônicos, eles são os CCDs (\textit{Charge-Couple Device}) e os \textit{Active-pixel sensors}, comumente chamados de sensores CMOS. Ambos os dispositivos são feitos com tecnologia MOS (\textit{Metal Oxide Semiconductor}), com CCDs baseados em capacitores MOS e os sensores CMOS baseados em amplificadores MOSFET. 
		
		Apesar dos sensores CCDs terem sido pioneiros no ramo de medição científica nos anos 80 e se tornado o sensor de escolha para praticamente toda aplicação de imagens nesta época, o desenvolvimento de sensores CMOS atingiu um ponto em que começou a substituir os sensores CCDs em aplicações de baixa performance. Com o passar do tempo os sensores de tecnologia CMOS se expandiram, enquanto os de tecnologia CCD estagnaram. A seguir entra-se em mais detalhes sobre cada tipo de sensor, suas características, vantagens e desvantagens.
		
	\section*{Tipo de sensores de imagem}
	\subsection*{Sensores CCD}
	\subsection*{Sensores CMOS}
	\subsection*{CCD vs. CMOS}
	\section*{Simulação de um sensor CMOS simplificado}
	\section*{Projeto de um sensor CMOS simplificado}
	\section*{Extra: Separação de cores}
	\section*{Conclusão}

\end{document}